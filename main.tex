%% Настройки документа
\documentclass[a4paper, 12pt] {article} %Параметры страницы  

\usepackage{array}
\newcolumntype{M}[1]{>{\centering\arraybackslash}m{#1}}
\newcolumntype{N}{@{}m{0pt}@{}}
\usepackage{float}

%% Стандартные пакеты
\usepackage {cmap} %Поиск в пдф
\usepackage [T2A] {fontenc} %Кодировка
\usepackage [utf8] {inputenc} %Кодировка исходного текста
\usepackage [english, russian] {babel} %Локализация и переносы

%% Кириллица в формулах
\usepackage{mathtext}

%% Математические пакеты 
\usepackage{amsmath,amsfonts,amssymb,amsthm,mathtools} % AMS
\usepackage{icomma} % Умная запятая 

%% Шрифты
\usepackage{euscript} % Шрифт Евклида
\usepackage{mathrsfs} % Матшрифт

\usepackage{extsizes}
\usepackage{fancyhdr}

\usepackage{graphicx}
\usepackage{setspace}

%% Графики
\usepackage{pgfplots}
\usepgfplotslibrary{fillbetween}
\usetikzlibrary{patterns}

\usepackage{geometry}
\geometry{top=20mm}
\geometry{bottom=24mm}
\geometry{left=16mm}
\geometry{right=18mm}

%% Листинг
\usepackage{listings}  
\usepackage{minted}
%% \usemintedstyle{emacs}

\usepackage[pdf]{graphviz}
\usepackage{morewrites}

\usepackage{psfrag}

\usepackage{xpatch}
\makeatletter
\newcommand*{\addFileDependency}[1]{% argument=file name and extension
  \typeout{(#1)}
  \@addtofilelist{#1}
  \IfFileExists{#1}{}{\typeout{No file #1.}}
}
\makeatother
\xpretocmd{\digraph}{\addFileDependency{#2.dot}}{}{}

\DeclareMathOperator{\Perm}{\mathop{Perm}}
\DeclareUnicodeCharacter{2212}{-}
\raggedbottom

\begin{document}

\thispagestyle{empty}

\begin{center}
	ФЕДЕРАЛЬНОЕ ГОСУДАРСТВЕННОЕ БЮДЖЕТНОЕ ОБРАЗОВАТЕЛЬНОЕ УЧРЕЖДЕНИЕ ВЫСШЕГО ОБРАЗОВАНИЯ НАЦИОНАЛЬНЫЙ ИССЛЕДОВАТЕЛЬСКИЙ УНИВЕРСИТЕТ "МЭИ"
\end{center}

\vspace{7cm}

\begin{center}
    \Huge Отчет

	\Large К домашней работе №3
	
	\Large По теоретическим моделям вычислений

	\Large Машины Тьюринга и квантовые вычисления
\end{center}

\vspace{3cm}

\begin{flushright}
	Выполнил
	
	Ушаков Н.А. 
	
	Студент группы А-05-19
\end{flushright}

\vfill
\begin{center}
	Москва 2022
\end{center}

\newpage
\section{Машины Тьюринга}

\subsection{Операции с числами}
Реализуйте машины Тьюринга, которые позволяют выполнять следующие операции:

\begin{enumerate}
    \item Сложение двух унарных чисел. 
    \\ \\
    Реализация:
    МТ принимает два числа, представляющих последовательность из <<1>>, разделенных знаком <<+>>. МТ проходит вправо по ленте, до знака <<+>>, заменяет его на <<1>>, складывая таким образом числа, и удаляет последнюю лишнюю <<1>>. 
    \\ \\
    Листинг \mintinline{console}{1_1.yaml}:
    
    \begin{minted}[xleftmargin=5pt, fontsize=\footnotesize]{yaml}
    input: '111+11'
    blank: ' ' 
    start state: to_right 
    table:
      to_right: # Движение вправо
        '1': R 
        '+': {write: '1', R: to_last} # Меняем + на 1 
      to_last: # Движение в конец
        '1': R
        ' ': {L: del_last}  
      del_last: # Удаляем последнюю 1
        '1': {write: ' ', R: done}
      done:
    \end{minted}
    
    Визуализация:
    \[\includegraphics[scale=0.5]{mt/1_1.png}\]
    
    \item Умножение унарных чисел. 
    \\ \\
    Реализация: 
    МТ принимает два числа, представляющих последовательность из <<1>>, разделенных знаком <<*>>. МТ последовательно уменьшает первое число и копирует второе на каждом шаге. 
    \\ \\
    Листинг \mintinline{console}{1_2.yaml}:
    \begin{minted}[xleftmargin=5pt, fontsize=\footnotesize]{yaml}
    input: '111*11'
    blank: ' '
    start state: each_first
    table:
      # Копируем second, first раз, уменьшая first, создавая каждый раз одну копию
      each_first: # Уменьшаем first на единицу 
        '1': {write: ' ', R: to_second}
        '*': {R: skip}
      to_second: # Движение к началу second 
        '1': R
        '*': {R: each_second} # Копирование
      next_first:
        ' ': {write: '1', R: each_first}
        ['1','*']: L
      skip: # Движение к результату
        '1': R
        ' ': {R: done}
      done:
      # Копирование second
      each_second: 
        ' ': {L: next_first}
        '1': {write: ' ', R: sep_r}
      sep_r:
        ' ': {R: add}
        '1': R
      add:
        ' ': {write: '1', L: sep_l}
        '1': R
      sep_l:
        ' ': {L: next_second}
        '1': L
      next_second:
        ' ': {write: '1', R: each_second}
        '1': L
    \end{minted}
    
    Визуализация:
    \[\includegraphics[scale=0.5]{mt/1_2.png}\]
    
    
\end{enumerate}

\subsection{Операции с языками и символами}
Реализуйте машины Тьюринга, которые позволяют выполнять следующие операции:

\begin{enumerate}
    \item Принадлежность языку $L = \{0^n1^n2^n\},n\ge 0$. 
    \\ \\
    Реализация: 
    МТ принимает слово, являющееся посл-тью вида <<0..1..2>> либо пустое слово. Проходясь по слову, МТ заменяет цифры <<0,1,2>> на <<x,y,z>> соответственно, что позволяет понять, рассматривалась ли уже данная цифра. Если, по окончании работы МТ, каретка оказалась на пустом символе, то совершается переход в состояние <<done>>, то есть слово принадлежит языку, иначе не принадлежит. 
    
    Листинг \mintinline{console}{2_1.yaml}:
    \begin{minted}[xleftmargin=5pt, fontsize=\footnotesize]{yaml}
    input: '001122'
    blank: ' '
    start state: start
    table:
      start:
        '0': {write: 'x', R: ch_one} # 0 на х
        ['y','z']: R
        ' ': {L: done}
      ch_one:
        ['0','y']: R
        '1': {write: 'y', R: ch_two} # 1 на y
      ch_two:
        ['0','1','z']: R
        '2': {write: 'z', L: ch_zero} # 2 на z
      ch_zero:
        ['0','1','y','z']: L
        'x': {R: start}
      done:
    \end{minted}
    
    Визуализация:
    \[\includegraphics[scale=0.5]{mt/2_1.png}\]
    
    \item Проверка соблюдения правильности скобок в строке (минимум 3 вида скобок).
    \\ \\
    Реализация: 
    МТ принимает проивзольную скобочную последовательность из элементов типа <<(,),\{,\},[,]>>. Каретка МТ движется вправо, до первой закрывающей скобки и, при встрече ее, заменяет <<(,[,\{>> на <<x,y,z>> соотвественно. После, осуществляется обратное движение влево, пока не будет найдена, соотвествующая открывающая скобка. Если встречена открывающая скобка другого типа или пустой символ, то скобочная последовательность неверная, совершается переход в состояние <<err>>. Алгоритм повторяется. Если, по окончании работы МТ, каретка оказалась на пустом символе, то совершается переход в <<done>>, последовательность была верной. 
    
    Листинг \mintinline{console}{2_2.yaml}:
    \begin{minted}[xleftmargin=5pt, fontsize=\footnotesize]{yaml}
    input: '()[()]{)()]}'
    blank: ' '
    start state: start
    table:
      start: # Движение вправо, до открытой скобки
        ['(','{','[','x','y','z']: R
        ')': {write: 'x', L: opn_rnd}
        ']': {write: 'y', L: opn_sqr}
        '}': {write: 'z', L: opn_fig}
        ' ': {L: to_last}
      opn_rnd: # Ищем открывающую круглую скобку
        [' ','[','{']: {R: err} 
        ['x','y','z',']','}']: L
        '(': {write: 'x', R: start}
      opn_sqr: # Ищем открывающую квадратную скобку
        [' ','(','{']: {R: err} 
        ['x','y','z',')','}']: L
        '[': {write: 'y', R: start}
      opn_fig: # Ищем открывающую фигурную скобку
        [' ','(','[']: {R: err} 
        ['x','y','z',')',']']: L
        '{': {write: 'z', R: start}
      to_last:
        ['(','[','{']: {R: err}
        ['x','y','z']: {R: done}
      done:
      err:
    \end{minted}
    
    Визуализация:
    \[\includegraphics[scale=0.5]{mt/2_2.png}\]
\end{enumerate}


\section{Квантовые вычисления}

\end{document}